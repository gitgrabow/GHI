\section{Input Parameters}

\subsection{Population}
\beq
\popDensity = \begin{pmatrix}{f_{ij}}\end{pmatrix},\quad i \in \{\mathrm{M},\mathrm{F}\}, j \in \{\mathrm{age\ classes}\}  
\eeq

$\mu^t_w$ is the population mean time walking, with $\mu^t_c$, $\mu^s_w$, $\mu^s_c$ defined similarly with speed ($s$), time ($t$), walking ($w$) and cycling ($c$).

\subsection{Relative Time and Speed for Walking and Cycling}

\beq
\relativeWalk = \begin{pmatrix}{{\rho^t_w}_{ij}}\end{pmatrix}
\eeq

as well as $\relativeCycle$, $\speedWalk$ and $\speedCycle$ defined likewise.

\section{Model}

See \textit{Baseline} worksheet.

\subsection{Mean Walking and Cycling Times (minutes)}

\begin{eqnarray}
%%\hat{\mathbf{M}}_w &=& \begin{pmatrix}{\mu_{ij}}\end{pmatrix}\\
%%\mu_{ij} &=& \frac{\mu_w^{0}}{\alpha}{\rho_w}_{ij}\\
\walkMean &=& \frac{\mu^t_w}{\alpha^t_w}\relativeWalk, \quad \alpha^t_w = \sum_{ij} f_{ij}{\rho^t_w}_{ij}\\
\cycleMean &=& \frac{\mu^t_c}{\alpha^t_c}\relativeCycle, \quad \alpha^t_c = \sum_{ij} f_{ij}{\rho^t_c}_{ij}
\end{eqnarray}

\subsection{Mean Walking and Cycling Speeds}

\begin{eqnarray}
%%\hat{\mathbf{M}}_w &=& \begin{pmatrix}{\mu_{ij}}\end{pmatrix}\\
%%\mu_{ij} &=& \frac{\mu_w^{0}}{\alpha}{\rho_w}_{ij}\\
\walkSpeedMean &=& \frac{\mu^s_w}{\alpha^s_w}\speedWalk, \quad \alpha^s_w = \sum_{ij} f_{ij}{\rho^s_w}_{ij}\\
\cycleSpeedMean &=& \frac{\mu^s_c}{\alpha^s_c}\speedCycle, \quad \alpha^s_c = \sum_{ij} f_{ij}{\rho^s_c}_{ij}
\end{eqnarray}


\subsection{METs}
\beq
\metTransport = \metWalk + \metCycle
\eeq

\beq
\metWalk = f_w\left(\speedWalk\right), \quad f_w\left( \sijw \right) = \max\{ \beta_0 + \beta_1 \sijw, 2.5 \}
\eeq

with $\beta_1 = 1.2216$, $\beta_0 = 0.0838$ 

\beq
\metCycle = \mathbf{6}
\eeq

\subsection{\textit{Relative} Mean Distance of Active Transport}

Not sure about the relative distance measure.  What are the units?  Is
it used later for road injury miodeling.  I don't think it is used
here.

\beq
\distWalkMean = \frac{1}{60} \speedWalk \cdot \walkMean
\eeq
\beq
\distCycleMean = \frac{1}{60} \speedCycle \cdot \cycleMean
\eeq

\subsection{Quintiles}

Estimate quintiles per group based on group mean $\mu_{ij}$ and
constant coefficient of variation, $\gamma$ using a with $T$
distributed as lognormal.  Divided the quintiles proportionally into
walking and cycling, based on within-group proportions.  Compute MET-hours per quintile as $\metWalk T_w$

And we arrive at quintiles for total MET-hours per week stratified by age group and sex, $\quintMat$.



\section{Relative Risk given two types of Exposure (Weekly Active Time and MET-hours)}

%% \begin{verbatim}
%% =IF(($'user page'.$N$39=0),C5^(1/C6^0.25),
%% IF(($'user page'.$N$39=1),C5^(1/C6^0.5),
%% IF(($'user page'.$N$39=2),C5^(1/C6^0.375),IF(($'user page'.$N$39=4),C5^(1/C6),
%% IF(($'user page'.$N$39=3),C5^(1/(LN(C6))),""
%% )))))
%% \end{verbatim}

Let $x$ be the ``exposure'' in \textit{Phy Activity RRs} and $\RR$ the ``RR'' in the same worksheet then the ``1 MET'' described in the worksheet is named $\RR_{\scriptsize{\textrm{MET}}}$ here.

$$\RRMET = {\RR}^{{x^{-k}}}$$ with $k$ a model parameter.

\subsection{Definitions of $\RR$, Exposure ($x$) and $\RRMET$}

Do not have these yet.

For developmental purposes we fix the vector $\RRvec$ to be
the vector of disease sprecific relative risks given an exposure of 1
MET-hour per week.

$\RRvec = \left( \RRMETi \right)$

\subsection{Disease Burden}

First we define a weight vector for disease burden, $\burdenVec$. With
$\RRvec$, $\quintMat$ and $\burdenVec$ we can compute measures of
total disease burden.

\subsubsection{\textit{GBDUS}}

What are these data?  It's a table look up as with Non-travel METs.
