The ITHIM model uses the comparative risk assesment framework to
estimate the population attributable fraction, $\af$.  In our case the
$\af$ represents the proportional change in national disease burden if
age and sex-specific active transport time means were changed.

\begin{equation}
\af = \frac{\int \! R(x)P(x) \, \mathrm{d}x  - \int \! R(x)Q(x) \, \mathrm{d}x}{\int \! R(x)P(x) \, \mathrm{d}x}
\end{equation}

Here $R(x)$ represents the risk when the individual has exposure $x$,
physical activity (\mets).  $P(x)$ and $Q(x)$ denote the population
density of individuals with exposure $x$, in the baseline and
alternate scenarios, respectively.  The exposure variable, $x$, is
modeled as the sum of two independent random variables, travel-related
exposure and non-travel-related exposure.  Non-travel-related exposure
is modeled with a \logNormal{} distribution with the age-sex specific
mean ratios for non-travel-related exposure are estimated from the
ATUS \cite{ATUS}.  The referent group mean is treated as a variable
and the coefficient of variation is treated as constant across age-sex
strata and estimated with ATUS.  We see in Figure \ref{dalyFigure}
that varying the mean values for non-travel-related exposure does not
greatly affect the estimate for disease burden.  The distribution for
travel-related exposure is estimated from the time spent walking or
cycling, i.e., active transport time, which as in the original ITHIM
model has a \logNormal{} distribution with constant coefficient of
variation across age-sex strata.  Travel-related exposure is estimated
from active transport time using assumptions about how much physical
activity is required for cycling and walking.
