We used the National Household Transportation Survey to estimate the
mean walk and cycle times for individuals that live in metropolitan
areas for each of seven states (CA, TX, NY, FL, VA, NC and AZ)
\cite{NHTS}.  These states were selected becaused they had sufficient
sample size for reasonable estimates of age-sex specific
travel-related walking and cycling means, Figure \ref{meanMatrices}.

Of these seven states one was found to show active transport behavior
that, if adopted nation-wide, would avert 75-100,000 DALYs overall.
Values for national disease-specific DALYs were found using the CDC
Wonder database \cite{CDCWonder}.  California stood out among the
estimates for national DALYs averted.  Most of the decrease in DALYs
is due to prevention of dementia in the oldest age group, $80+$ years
old.  California exhibits much greater walking and cycling means in
this age group than the national average. In particular we see the
cycling mean We also see prevention of depression and diabetes in the
middle age groups due to increased active transport time.
