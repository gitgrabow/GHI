Regular exercise has been shown to provide health benefits in terms of
chronic disease prevention \cite{warburton2006}, cognitive function
\cite{hillman2008}, and overall well being and satisfaction with life
\cite{maher2013}. Still only half of Americans obtain the recommended
levels of aerobic activity, which is 2.5 hours per week. There is
considerable variability in the attainment of this goal by location
within the country. Supportive local built and policy environments are
important determinants of population-level activity
patterns. Particularly, as they facilitate and foster routine moderate
physical activity, such as walking, local decissions on design and
policy can play a huge role in the Nation's health.

To better understand the potential for attainable shifts in
transportation behavior across the US, we employed a comparative risk
assessment methodology to compare national averages for time spent
walking and cycling for transport to those of specific states. Our
estimates can provide an important evidence base to encourage new
construction, retrofits, land use policy, and programatic funding to
support active transportation. Particularly in places with poor
attainmnet of the physical activity guidelines, such investments can
be importnat pieces of lareger population health interventions that
have effective and long-lasting impacts.

ITHIM is a statistical model that integrates data on travel patterns,
physical activity, fine particulate matter, GHG emissions, and disease
and injuries. Based on population and travel scenarios. The model has
been used to calculate the health impacts of walking and bicycling
short distances usually traveled by car or driving low-emission
automobiles \cite{woodcock2013,maizlish2013}.

The ITHIM model uses comparative risk assessment through which it
formulates a change in the disease burden, resulting from the shift in
the exposure distribution from a baseline scenario to an alternative
scenario.
