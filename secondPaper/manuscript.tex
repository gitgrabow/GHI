\documentclass{bioinfo}
\usepackage{url}
\copyrightyear{2016}
\pubyear{2016}

\newcommand{\col}[2][red]{\textcolor{#1}{#2}}
\newcommand{\bi}{\begin{itemize}}
\newcommand{\ei}{\end{itemize}}
\newcommand{\package}{\emph{ITHIM}}
\newcommand{\ithim}{\emph{ITHIM}}
\newcommand{\R}{\emph{R}}
\newcommand{\sgy}[1]{{\itshape\col{#1}}}
\newcommand{\XX}{\sgy{XX}}
\newcommand{\webpage}{\url{http://www.ithim.ghi.wisc.edu}}
\newcommand{\af}{\mathrm{PAF}}
\newcommand{\mets}{MET-hrs./week}
\newcommand{\pEm}[1]{$\mathrm{PM}_{#1}$}
\newcommand{\vmt}{$\mathrm{VMT}$}
\newcommand{\logNormal}{log-normal}


\begin{document}
\firstpage{1}

\title[]{A comparative risk assessment of U.S. regional active
  transport using the ITHIM model}
\author[Younkin \textit{et~al}]{Samuel G. Younkin$^{1}$,
  Jason Vargo$^{1}$\footnote{to whom correspondence should be addressed},
  $\ldots$
  \ and Jonathan Patz$^{1}$}
\address{$^{1}$Global Health Institute\\
University of Wisconsin{\textendash}Madison, Madison, WI USA\\
}
\history{Received on XXXXX; revised on XXXXX; accepted on XXXXX}

\editor{Associate Editor: XXXXXXX}

\maketitle

\begin{abstract}
\section{Summary:}

\section{Contact:}
\href{javargo@wisc.edu}{javargo@wisc.edu}


\end{abstract}

\textit{The Americal Journal of Public Health} (AJPH) states the following:

\begin{quote}
There are 12 submission categories: Research Articles, Brief
Articles, Systematic Reviews, Letters and Responses, The Editors
Choice, Opinion Editorials, Commentaries, Analytic Essays, History
Essays, Voices, News, and Images.

Research Articles report the results of original public health
research in up to 3500 words in the text, a structured abstract, up to
4 tables and figures, and no more than 35 references. The text must
have an introduction and separate sections for Methods, Results,
Discussion, and, Public Health Implications. This format is the
highest priority for AJPH and represents the majority of papers
published.

Brief Articles present preliminary findings or novel findings in up to
1200 words in the main text, a structured (except if justified
otherwise in the cover letter) abstract, up to 1 table or figure, and
no more than 12 references.  Research Brief Articles must have an
introduction and separate sections for the Methods, Results,
Discussion, and Public Health Implications. Some policy -focused Brief
Articles which are shor t essays and do not report study results do
not require the “method, results, discussion, public health
implications” format subheadings.
\end{quote}

\section{Introduction}
Hello world.


\section{Methods}
\begin{methods}
The ITHIM model uses the comparative risk assesment framework to
estimate the population attributable fraction, $\af$.  In our case the
$\af$ represents the proportional change in national disease burden if
age and sex-specific active transport time means were changed.

\begin{equation}
\af = \frac{\int \! R(x)P(x) \, \mathrm{d}x  - \int \! R(x)Q(x) \, \mathrm{d}x}{\int \! R(x)P(x) \, \mathrm{d}x}
\end{equation}

Here $R(x)$ represents the risk when the individual has exposure $x$,
physical activity (\mets).  $P(x)$ and $Q(x)$ denote the population
density of individuals with exposure $x$, in the baseline and
alternate scenarios, respectively.  The exposure variable, $x$, is
modeled as the sum of two independent random variables, travel-related
exposure and non-travel-related exposure.  Non-travel-related exposure
is modeled with a \logNormal{} distribution with the age-sex specific
mean ratios for non-travel-related exposure are estimated from the
ATUS \cite{ATUS}.  The referent group mean is treated as a variable
and the coefficient of variation is treated as constant across age-sex
strata and estimated with ATUS.  We see in Figure \ref{dalyFigure}
that varying the mean values for non-travel-related exposure does not
greatly affect the estimate for disease burden.  The distribution for
travel-related exposure is estimated from the time spent walking or
cycling, i.e., active transport time, which as in the original ITHIM
model has a \logNormal{} distribution with constant coefficient of
variation across age-sex strata.  Travel-related exposure is estimated
from active transport time using assumptions about how much physical
activity is required for cycling and walking.

\end{methods}

\section{Results}
We used the National Household Transportation Survey to estimate the
mean walk and cycle times for individuals that live in metropolitan
areas for each of seven states (CA, TX, NY, FL, VA, NC and AZ)
\cite{NHTS}.  These states were selected becaused they had sufficient
sample size for reasonable estimates of age-sex specific
travel-related walking and cycling means, Figure \ref{meanMatrices}.

Of these seven states one was found to show active transport behavior
that, if adopted nation-wide, would avert 75-100,000 DALYs overall.
Values for national disease-specific DALYs were found using the CDC
Wonder database \cite{CDCWonder}.  California stood out among the
estimates for national DALYs averted.  Most of the decrease in DALYs
is due to prevention of dementia in the oldest age group, $80+$ years
old.  California exhibits much greater walking and cycling means in
this age group than the national average. In particular we see the
cycling mean We also see prevention of depression and diabetes in the
middle age groups due to increased active transport time.


\section{Discussion}
Nulla mi mi, venenatis sed ipsum varius, Table~\ref{table1} volutpat
euismod diam. Proin rutrum vel massa non gravida. Quisque tempor sem
et dignissim rutrum. Lorem ipsum dolor sit amet, consectetur
adipiscing elit. Morbi at justo vitae nulla elementum commodo eu id
massa. In vitae diam ac augue semper tincidunt eu ut eros. Fusce
fringilla erat porttitor lectus cursus, vel sagittis arcu
lobortis. Aliquam in enim semper, aliquam massa id, cursus
neque. Praesent faucibus semper libero~\cite{bib3}.


\section{Public Health Implications}
\input{./tex/publicHealthImplications}

\section*{Acknowledgment}
\paragraph{Funding\textcolon} SGY is supported by \ldots,\ JV, JP by \ldots

%\bibliographystyle{natbib}
\bibliographystyle{bmc_article}
%\bibliographystyle{mypapers}
\bibliography{../tex/ITHIM}
\end{document}

% \subsection{Active Transport}
% Regular exercise has been shown to provide health benefits in terms of
% chronic disease prevention \cite{warburton2006}, cognitive function
% \cite{hillman2008}, and overall well being and satisfaction with life
% \cite{maher2013}. Still only half of Americans obtain the recommended
% levels of aerobic activity, which is 2.5 hours per week. There is
% considerable variability in the attainment of this goal by location
% within the country. Supportive local built and policy environments are
% important determinants of population-level activity
% patterns. Particularly, as they facilitate and foster routine moderate
% physical activity, such as walking, local decissions on design and
% policy can play a huge role in the Nation's health.

% To better understand the potential for attainable shifts in
% transportation behavior across the US, we employed a comparative risk
% assessment methodology to compare national averages for time spent
% walking and cycling for transport to those of specific states. Our
% estimates can provide an important evidence base to encourage new
% construction, retrofits, land use policy, and programatic funding to
% support active transportation. Particularly in places with poor
% attainmnet of the physical activity guidelines, such investments can
% be importnat pieces of lareger population health interventions that
% have effective and long-lasting impacts.

% ITHIM is a statistical model that integrates data on travel patterns,
% physical activity, fine particulate matter, GHG emissions, and disease
% and injuries. Based on population and travel scenarios. The model has
% been used to calculate the health impacts of walking and bicycling
% short distances usually traveled by car or driving low-emission
% automobiles \cite{woodcock2013,maizlish2013}.

% The ITHIM model uses comparative risk assessment through which it
% formulates a change in the disease burden, resulting from the shift in
% the exposure distribution from a baseline scenario to an alternative
% scenario.
% The ITHIM model uses the comparative risk assesment framework to
% estimate the population attributable fraction, $\af$.  In our case the
% $\af$ represents the proportional change in national disease burden if
% age and sex-specific active transport time means were changed.

% \begin{equation}
% \af = \frac{\int \! R(x)P(x) \, \mathrm{d}x  - \int \! R(x)Q(x) \, \mathrm{d}x}{\int \! R(x)P(x) \, \mathrm{d}x}
% \end{equation}

% Here $R(x)$ represents the risk when the individual has exposure $x$,
% physical activity (\mets).  $P(x)$ and $Q(x)$ denote the population
% density of individuals with exposure $x$, in the baseline and
% alternate scenarios, respectively.  The exposure variable, $x$, is
% modeled as the sum of two independent random variables, travel-related
% exposure and non-travel-related exposure.  Non-travel-related exposure
% is modeled with a \logNormal{} distribution with the age-sex specific
% mean ratios for non-travel-related exposure are estimated from the
% ATUS \cite{ATUS}.  The referent group mean is treated as a variable
% and the coefficient of variation is treated as constant across age-sex
% strata and estimated with ATUS.  We see in Figure \ref{dalyFigure}
% that varying the mean values for non-travel-related exposure does not
% greatly affect the estimate for disease burden.  The distribution for
% travel-related exposure is estimated from the time spent walking or
% cycling, i.e., active transport time, which as in the original ITHIM
% model has a \logNormal{} distribution with constant coefficient of
% variation across age-sex strata.  Travel-related exposure is estimated
% from active transport time using assumptions about how much physical
% activity is required for cycling and walking.
% \begin{figure}[t]
%     \centerline{\includegraphics[width=0.5\textwidth]{./figures/fig2}}
%     \caption{}\label{dalyFigure}
% \end{figure}

% \begin{figure}[t]
%   \centerline{\includegraphics[width=0.5\textwidth]{./figures/fig3.pdf}}
%     \caption{Mean travel time by mode (walking/cycling) for each of
%       seven states.  National means are displayed on the
%       right.}\label{meanMatrices}
% \end{figure}

% \begin{figure}[t]
%   \centerline{\includegraphics[width=0.5\textwidth]{./figures/fig4.pdf}}
%     \caption{}\label{California}
% \end{figure}

% We used the National Household Transportation Survey to estimate the
% mean walk and cycle times for individuals that live in metropolitan
% areas for each of seven states (CA, TX, NY, FL, VA, NC and AZ)
% \cite{NHTS}.  These states were selected becaused they had sufficient
% sample size for reasonable estimates of age-sex specific
% travel-related walking and cycling means, Figure \ref{meanMatrices}.

% Of these seven states one was found to show active transport behavior
% that, if adopted nation-wide, would avert 75-100,000 DALYs overall.
% Values for national disease-specific DALYs were found using the CDC
% Wonder database \cite{CDCWonder}.  California stood out among the
% estimates for national DALYs averted.  Most of the decrease in DALYs
% is due to prevention of dementia in the oldest age group, $80+$ years
% old.  California exhibits much greater walking and cycling means in
% this age group than the national average. In particular we see the
% cycling mean We also see prevention of depression and diabetes in the
% middle age groups due to increased active transport time.
