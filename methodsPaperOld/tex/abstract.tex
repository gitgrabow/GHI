\section{Summary:}

While increased active transportation, primarily cycling and walking,
is often held up as a goal for improved public health --- via reduced
chronic disease, improved air quality, and reductions in greenhouse
gas emissions — obtaining estimates for the quantiative impacts of
such shifts remains infrequent, if not altogether absent from related
policy action. Using U.S. metropolitan counties from seven states we
perform a comparative risk analysis between each regions and national
averages as the baseline.  We employ a portion of the Integrated
Transport and Health Impacts Model (ITHIM) to investigate the reduced
chronic disease burden of ($x$, $y$, $z$ outcomes) from changing
levels of physical activity with population shifts in the amount of
biking and walking done for transport. The analysis is informed by
regional travel parameters from the National Household Transportation
Survey, non-travel activity metrics from the American Time Use Survey,
and baseline disease burden from the Centers for Disease Control and
Prevention's WONDER database and the Global Burden of Disease dataset.
Our findings show that the California and New York State have travel
patterns that, if mirrored by the nation, would save the U.S. upwards
of fifty thousand disability adjusted life years.

%% \section{Availability:}

%% %% \package{} may be explored using the user interface avaiable
%% %% at \webpage{}.
%% The source code and data files for \package{} are publicly available
%% on GitHub via a link on \webpage{}.

\section{Supplementary information:}

See \webpage{} for links to the source code, manual, tutorial and
links for more information about the ITHIM model.

\section{Contact:}
\href{javargo@wisc.edu}{javargo@wisc.edu}

