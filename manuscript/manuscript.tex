\documentclass{bioinfo}
\usepackage{url}
\copyrightyear{2016}
\pubyear{2016}

\newcommand{\col}[2][red]{\textcolor{#1}{#2}}
\newcommand{\bi}{\begin{itemize}}
\newcommand{\ei}{\end{itemize}}
\newcommand{\package}{\emph{ITHIM}}
\newcommand{\ithim}{\emph{ITHIM}}
\newcommand{\R}{\emph{R}}
\newcommand{\sgy}[1]{{\itshape\col{#1}}}
\newcommand{\XX}{\sgy{XX}}
\newcommand{\webpage}{\url{http://www.ithim.ghi.wisc.edu}}
\newcommand{\af}{\rho}
\newcommand{\mets}{MET-hrs./week}
\newcommand{\pEm}[1]{$\mathrm{PM}_{#1}$}
\newcommand{\vmt}{$\mathrm{VMT}$}

\begin{document}
\firstpage{1}

\title[\package{}]{A comparative risk assesment of U.S. regional
  active transport using the ITHIM model}
\author[Younkin \textit{et~al}]{Samuel G. Younkin$^{1}$,
  Jason Vargo$^{1}$\footnote{to whom correspondence should be addressed},
  $\ldots$
  \ and Jonathan Patz$^{1}$}
\address{$^{1}$Global Health Institute\\
University of Wisconsin{\textendash}Madison, Madison, WI USA\\
}
\history{Received on XXXXX; revised on XXXXX; accepted on XXXXX}

\editor{Associate Editor: XXXXXXX}

\maketitle

\begin{abstract}

\section{Summary:}
\col[blue]{While increased active transportation, primarily cycling
  and walk, is often held up as a goal for improved public health —
  via reduced chronic disease, improved air quality, and reductions in
  greenhouse gas emissions --- obtaining estimates for the quantiative
  impacts of such shifts remains infrequent, in not altogether
  absent. Using U.S. metropolitan counties from the 10 regions defined
  by the Department of Health and Human Services (HHS1-10) we perform
  a comparative risk analysis between each regions and national
  averages as the baseline.  We investigated the reduced chronic
  disease burden of ($x$, $y$, $z$ outcomes) from increased physical
  activity from population shifts in the amount of biking and walking
  doen for transport.  Our methods employ the Integrated Transport and
  Health Impacts Model (ITHIM) informed by regional travel parameters
  using the National Household Transportation Survey, non-travel
  activity metircs from the American Time Use Survey, and baseline
  Disease burden from the Centers for Disease Control and Prevention's
  WONDER database.  Our findings show that the West coast and the
  Northeast of the U.S. have travel patterns that, if mirrored by the
  nation, would save the U.S. thousands to hundreds of thousands
  DALYs.}

\section{Availability:}
\col[blue]{\package{} may be explored using the user interface
  avaiable at \webpage{}.  The source code and data files for
  \package{} are publicly available on GitHub at \webpage{}.}

\section{Supplementary information:}

\col[blue]{See \webpage{} for links to the source code, manual,
  tutorial and links for more information about the ITHIM model.}

\section{Contact:}
\col[blue]{\href{javargo@wisc.edu}{javargo@wisc.edu}}
\end{abstract}


\section{Introduction}

\col{ITHIM is a statistical model that integrates data on travel
  patterns, physical activity, fine particulate matter, GHG emissions,
  and disease and injuries. Based on population and travel
  scenarios. The model has been used to calculate the health impacts
  of walking and bicycling short distances usually traveled by car or
  driving low-emission automobiles \cite{woodcock2013,maizlish2013}.}

\col{The ITHIM model uses comparative risk assessment through which it
  formulates a change in the disease burden, resulting from the shift
  in the exposure distribution from a baseline scenario to an
  alternative scenario.}

\col{Lorem ipsum dolor sit amet, consectetur adipisicing elit, sed do
  eiusmod tempor incididunt ut labore et dolore magna aliqua. Ut enim
  ad minim veniam, quis nostrud exercitation ullamco laboris nisi ut
  aliquip ex ea commodo consequat. Duis aute irure dolor in
  reprehenderit in voluptate velit esse cillum dolore eu fugiat nulla
  pariatur. Excepteur sint occaecat cupidatat non proident, sunt in
  culpa qui officia deserunt mollit anim id est laborum. Lorem ipsum
  dolor sit amet, consectetur adipisicing elit, sed do eiusmod tempor
  incididunt ut labore et dolore magna aliqua.}

\col{Lorem ipsum dolor sit amet, consectetur adipisicing elit, sed do
  eiusmod tempor incididunt ut labore et dolore magna aliqua. Ut enim
  ad minim veniam, quis nostrud exercitation ullamco laboris nisi ut
  aliquip ex ea commodo consequat. Duis aute irure dolor in
  reprehenderit in voluptate velit esse cillum dolore eu fugiat nulla
  pariatur. Excepteur sint occaecat cupidatat non proident, sunt in
  culpa qui officia deserunt mollit anim id est laborum.}


\begin{methods}

\section{Methods}

\col[blue]{\textit{
    The ITHIM model uses the comparative risk
    assesment framework to estimate the population attributable
    fraction, $\af$.  In our case $\af$ represents the proportional
    change in mortality and morbidity if the physical activity levels
    were adjusted to represent an alternate scenario.
}}

\begin{equation}
\af = \frac{\int R(x)P(x) dx  - \int R(x)Q(x) dx }{\int R(x)P(x) dx}
\end{equation}

\col[blue]{\textit{
    Here $R(x)$ represents the risk when the individual has exposure $x$,
in our case physical activity (\mets).  $P(x)$ and $Q(x)$ denote the
population density of individuals with exposure $x$, in the baseline
and alternate scenarios, respectively.  The exposure variable, $x$, is
modeled as the sum of travel-related exposure and non-travel-related
exposure.
}}

\subsection{Active Transport}

\col[blue]{\textit{The time spent walking or cycling, i.e., active
    transport time, is modeled as a log-normal random variable with
    means defined for each age-sex stratum.  The non-travel-related
    physical activity is also modeled as log-normal with
    stratum-specific means.}}

\col[blue]{\textit{
The travel-related exposure is estimated using active transport time
along with assumptions about how much physical activity is required
for cycling and walking.  Active transport time and non-travel-related
physical activity are modeled as independent random variables.  A
fixed value was used as the coefficient of variation for the
distributin of active transport time. A different, but also fixed,
value was used for the non-travel-related physical activity
distribution.
}}

\subsection{Road Injuries}

\col[red]{\textit{
We chose to model road injuries in a more simplifed manner than the
authors of ITHIM.  The original ITHIM requires information about the
transportational infrastructure that cannot be accurately modeled on a
regional level, and is more well-suited to say a collection of
counties as a case study. \cite{maizlish2013}  We use the overall percent
of travel by walking and road fatalities per U.S. metropolitan area to
estimate road fatalaties in U.S. metropolitan areas given the percent
of travel by walking. \cite{UWTechReport}
}}

\col[red]{\textit{
We need to estimate the increase/decrease in road injuries/fatalaties
nation-wide as walking/cycling time increases.  We need a relative
risk of road fatalaty as a function of active transport time...
}}

\subsection{Air Pollution}

\col[red]{\textit{
    Can we model the decrease in \pEm{2.5} nationally as a function of
decrease in driving time?  The original plan was to use a baseline
value for \pEm{2.5} and a linear relationship bewteen change in
\pEm{2.5} and change in \vmt{}.
}}

\end{methods}

\section{U.S. Metro-Regional Analysis}

\begin{figure}[t]
    \centerline{\includegraphics[width=0.4\textwidth]{./figures/fig3}}
    \caption{Foo}\label{meanMatrices}
\end{figure}

\col[blue]{\begin{itemize}
\item We used the National Household Transportation Survey \cite{NHTS} to
  estimate the mean walk and cycle times for individuals that live in
  metropolitan areas ($\textrm{DAYV2PUB.MSA} \in (1,2,\ldots,5)$ which
  corresponds to individuals whose homes are in an MSA or cMSA)
  \bi\item Data quality issues along with assumptions about travel speed
created mirky data.  The median values of trip time were used to avoid
undue influence by outliers.\ei
\item We then divided these into the 10 regions of the U.S. defined by
  the Department of Health and Human Services (HH1-10)
\item Using only the active transport component of the ITHIM model we
  ran comparative analyses for each of the ten regions, using the
  national averages as the baseline.
\item We used data from the American Time Use Survey \cite{ATUS} to
  estimate the relative means for each of the age-sex strata.  We then
  treat the mean of the referent group as a variable and compute
  results over a grid of values for this variable.  See Figure
  \ref{dalyFigure}.
\item Values for DALYs were found using the CDC Wonder database \cite{CDCWonder}
\end{itemize}}


\begin{figure}[t]
    \centerline{\includegraphics[width=0.4\textwidth]{./figures/HHSmap}}
    \centerline{\includegraphics[width=0.4\textwidth]{./figures/fig2}}
    \caption{Change in total national DALYs for each regional
      transportation estimates.}\label{dalyFigure}
\end{figure}

\section*{Acknowledgment}
\paragraph{Funding\textcolon} \col[red]{SGY is supported by$\ldots$ , JV, JP, $\ldots$}

%\bibliographystyle{natbib}
\bibliographystyle{bmc_article}
%\bibliographystyle{mypapers}
\bibliography{ITHIM}
\end{document}
