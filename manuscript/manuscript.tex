\documentclass{bioinfo}
\usepackage{url}
\copyrightyear{2016}
\pubyear{2016}

\newcommand{\col}[2][red]{\textcolor{#1}{#2}}
\newcommand{\bi}{\begin{itemize}}
\newcommand{\ei}{\end{itemize}}
\newcommand{\package}{\emph{ITHIM}}
\newcommand{\compendium}{\emph{ReadSimCompendium}}
\newcommand{\ithim}{\emph{ITHIM}}
\renewcommand{\bowtie}{\emph{bowtie}}
\newcommand{\ebseq}{\emph{EBSeq}}
\newcommand{\deseq}{\emph{DESeq2}}
\newcommand{\R}{\emph{R}}
\newcommand{\rnaseq}{RNA-seq}
\newcommand{\gi}{gene/isoform}
\newcommand{\gis}{genes/isoforms}
\newcommand{\pest}{\texttt{ParamEstimation}}
\newcommand{\csimu}{\texttt{CountSimu}}
\newcommand{\writeRSEM}{\texttt{writeRSEMFiles}}
\newcommand{\sgy}[1]{{\itshape\col{#1}}}
\newcommand{\XX}{\sgy{XX}}

\begin{document}
\firstpage{1}

\title[\package{}]{\package{}: an \R{} package for implemetation of
  the ITHIM model}
\author[Younkin \textit{et~al}]{Samuel G. Younkin$^{1}$,
  Jason Vargo$^{1}$\footnote{to whom correspondence should be addressed},
  $\ldots$
  \ and Jonathan Patz$^{1}$}
\address{$^{1}$Global Health Institute\\
University of Wisconsin{\textendash}Madison, Madison, WI USA\\
}
\history{Received on XXXXX; revised on XXXXX; accepted on XXXXX}

\editor{Associate Editor: XXXXXXX}

\maketitle

\begin{abstract}

\section{Summary:}

Lorem ipsum dolor sit amet, consectetur adipisicing elit, sed do
eiusmod tempor incididunt ut labore et dolore magna aliqua. Ut enim ad
minim veniam, quis nostrud exercitation ullamco laboris nisi ut
aliquip ex ea commodo consequat. Duis aute irure dolor in
reprehenderit in voluptate velit esse cillum dolore eu fugiat nulla
pariatur. Excepteur sint occaecat cupidatat non proident, sunt in
culpa qui officia deserunt mollit anim id est laborum. Lorem ipsum
dolor sit amet, consectetur adipisicing elit, sed do eiusmod tempor
incididunt ut labore et dolore magna aliqua. Ut enim ad minim veniam,
quis nostrud exercitation ullamco laboris nisi ut aliquip ex ea
commodo consequat. Duis aute irure dolor in reprehenderit in voluptate
velit esse cillum dolore eu fugiat nulla pariatur. Excepteur sint
occaecat cupidatat non proident, sunt in culpa qui officia deserunt
mollit anim id est laborum. Lorem ipsum dolor sit amet, consectetur
adipisicing elit, sed do eiusmod tempor incididunt ut labore et dolore
magna aliqua. Ut enim ad minim veniam, quis nostrud exercitation
ullamco laboris nisi ut aliquip ex ea commodo consequat.

\section{Availability:}
The source code and data files for \package{} are publicly available
on GitHub at \url{https://github.com/syounkin/ITHIM}.

\section{Supplementary information:}

\url{http://www.ithim.ghi.wisc.edu}.

\section{Contact:}
\href{patz@wisc.edu}{patz@wisc.edu}
\end{abstract}


\section{Introduction}

ITHIM is a mathematical model that integrates data on travel patterns,
physical activity, fine particulate matter, GHG emissions, and disease
and injuries. Based on population and travel scenarios. The model has
been used to calculate the health impacts of walking and bicycling
short distances usually traveled by car or driving low-emission
automobiles \cite{woodcock2013,maizlish2013}.



\begin{methods}
\section{Method}

The ITHIM model uses comparative risk assessment through which it
formulates a change in the disease burden, resulting from the shift in
the exposure distribution from a baseline scenario to an alternative
scenario.

\subsection{Data Sources}

\subsection{U.S. Metro-Regional Analysis}

Lorem ipsum dolor sit amet, consectetur adipisicing elit, sed do
eiusmod tempor incididunt ut labore et dolore magna aliqua. Ut enim ad
minim veniam, quis nostrud exercitation ullamco laboris nisi ut
aliquip ex ea commodo consequat. Duis aute irure dolor in
reprehenderit in voluptate velit esse cillum dolore eu fugiat nulla
pariatur. Excepteur sint occaecat cupidatat non proident, sunt in
culpa qui officia deserunt mollit anim id est laborum. Lorem ipsum
dolor sit amet, consectetur adipisicing elit, sed do eiusmod tempor
incididunt ut labore et dolore magna aliqua. Ut enim ad minim veniam,
quis nostrud exercitation ullamco laboris nisi ut aliquip ex ea
commodo consequat. Duis aute irure dolor in reprehenderit in voluptate
velit esse cillum dolore eu fugiat nulla pariatur. Excepteur sint
occaecat cupidatat non proident, sunt in culpa qui officia deserunt
mollit anim id est laborum. Lorem ipsum dolor sit amet, consectetur
adipisicing elit, sed do eiusmod tempor incididunt ut labore et dolore
magna aliqua. Ut enim ad minim veniam, quis nostrud exercitation
ullamco laboris nisi ut aliquip ex ea commodo consequat.


\begin{figure}[t]
    \centerline{\includegraphics[width=0.4\textwidth]{./figures/hhsmap}}\centerline{\includegraphics[width=0.4\textwidth]{./figures/ITHIMObjs-2}}
    \caption{Change in total national DALYs for each regional transportation
estimates.
}\label{dalyFigure}
\end{figure}
\end{methods}
\section*{Acknowledgment}

\paragraph{Funding\textcolon} SGY is supported by$\ldots$ , JV, JP, $\ldots$

%\bibliographystyle{natbib}
\bibliographystyle{bmc_article}
%\bibliographystyle{mypapers}
\bibliography{ITHIM}
\end{document}

%~~~~~~~~~~~~~~~~~~~~~~~~~~~~~~~~~~~~~~~~~~~~~~~~~~~~~~~~~~~~~~~~~~
%~~~~~~~~~~~~~~~~~~~~~~~~~~~~~~~~~~~~~~~~~~~~~~~~~~~~~~~~~~~~~~~~~~
%~~~~~~~~~~~~~~~~~~~~~~~~~~~~~~~~~~~~~~~~~~~~~~~~~~~~~~~~~~~~~~~~~~
%\bibliographystyle{achemnat}
%\bibliographystyle{plainnat}
%\bibliographystyle{abbrv}
%\bibliographystyle{bioinformatics}
%
%\bibliographystyle{plain}
%
% RNA-seq is a powerful technology used to measure genome-wide
% transcription and dozens of statistical methods have been developed
% recently to facilitate its use, including methods for estimating
% expression and identifying differential expression (DE)
% \cite{Oshlack2010}. The ability to evaluate these methods and compare
% their performance in various settings is essential to ensure that
% reliable information is obtained. The Microarray Quality Control
% (MAQC) \cite{shi2006} data sets have proven useful toward this
% end. However, they are limited by small sample sizes and old
% protocol. Evaluation of DE analysis methods can be performed with
% simulated data, and a handful of methods to simulate RNA-seq reads
% have been developed including FluxSimulator and BEERS
% \cite{Griebel2012, Grant2011}.  BEERS is slower than \package{} and,
% as with FluxSimulator, does not allow specification of read count
% means and are therefore not useful for DE-based read simulation.

% We have developed \package{} to simulate reads efficiently under a
% realistic model of DE across two treatment groups, and to use these
% reads to estimate expression levels for downstream analyses, all
% within the \R{} environment. \package{} is built on the software
% package \rsem{} used primarily for transcript quantification
% \cite{li2011}.  To induce differential expression of transcripts we
% adjust the relative transcript abundance across treatment groups in
% the empirical model used by the \rsem{} simulator.  The reads will be
% simulated under the same design as the sample data.  If the sample
% data are paired-end then the simulate reads will also be paired-end.

% Transcript abundances vary substantially across \rnaseq{} experiments.
% Factors such as cell type, organism, and experimental protocol affect
% the expected transcript abundances.  \package{} builds its model with
% sample data from an \rnaseq{} experiment, and uses the empirical read
% count distributions to select a realistic distribution for mean and
% dispersion parameters.  The total expression and fold change
% relationship for the simulated data is also modeled from these data
% and fold change levels are selected to mimic this
% relationship. \package{} also takes advantage of the transcript
% quantification provided by \rsem{} by allowing the user to estimate
% read counts from the simulated reads.  Both the read-based simulation
% and differential expression analysis may be performed within \R{}
% using any of the many \R{} packages available for DE analysis, e.g.,
% DESeq2, EdgeR, EBSeq \cite{love2014, Robinson2010, leng2013}.
% \package{} implements a two stage simulation.  First, it simulates
% read counts per transcript under a negative binomial distribution
% using empirical mean and dispersion estimates.  Second, it calculates
% relative transcript abundance using estimates of transcript length and
% read-count, updates the sample model, and executes the \rsem{} read
% simulator.

% In the first step, read counts for the untreated group are drawn from
% a negative binomial distribution with mean and dispersion parameters
% $\hat{\mu_0}$, $\hat{\varphi_0}$ estimated from the untreated
% group. In the treated group, differentially expressed transcripts are
% simulated using $\mu_1 = \gamma\mu_0$, $\varphi_1=\varphi_0$.  The
% parameter $\gamma$ is chosen to mimic the fold-change vs.\ total
% expression relationship observed empirically, see Figure
% \ref{ma-transcript}.  \sgy{Observed fold-change values decrease with
% expression level.  If fold-change was assigned independently of
% expression level then transcripts with high levels of expression would
% be assigned\ldots}

% Fold change among DE transcripts is chosen by binning the transcripts
% by overall expression into equally sized bins, and using the bin-wise
% standard deviation of the log fold-change to construct an appropriate
% interval for the fold-change, from which a value is sampled randomly.
% The interval is defined as $I_{\textrm{bin}} =
% (k_1\sigma_{\textrm{bin}}, k_2\sigma_{\textrm{bin}})$, where
% $\sigma_{\textrm{bin}}$ is the bin-wise standard deviation of log
% fold-change, and $k_1$ and $k_2$ are chosen by the user.  In the
% following chromosome 22 example, values of $k_1=?$ and $k_2=?$ were
% used with $B=?$ bins.  By tuning parameters $k_1$ and $k_2$ the user
% can control the signal strength while preserving the relationship
% between fold-change and total expression. \sgy{\package{} could
%   easily be adapted to allow user-defined values for $\gamma$, $\mu_i$
%   and $\varphi_i$.}

% Three files are needed for \package{}; a transcript template, the
% read-count estimates for all samples, and a sample model file.  All of
% these are easily created using \rsem{} utilities. A GTF file,
% transcript-to-gene file such as ``knownIsoforms.txt'' from the UCSC
% Genome Browser, and a FASTA reference sequence, or a Multi-FASTA file
% containing transcript sequences, may be used to create the reference
% files.

% In the second stage, the simulated read counts are used to compute the
% relative transcript abundance needed for the read simulation.
% \package{} can return either the gene or isoform level estimated read
% counts from \rsem{}, or simply write the fastq read file.  The
% software package \bowtie{} is used for the alignment
% \cite{langmead2009}.

% We simulated 1,000,000 reads for each of ten samples from two
% treatment groups, and performed DE analysis on all transcripts from
% chromosome 22 using \ebseq{} \cite{leng2013}.  One hundred genes (288
% transcripts) were selected at random to be DE.  In the top of Figure
% \ref{ma-transcript} the log fold-change is plotted versus total
% expression given the simulated counts.  Beneath it is the same figure,
% bottom of Figure \ref{ma-transcript}, but using the estimate counts of
% the simulated reads.  The reads are simulated in such a way as to
% preserve the relationship between fold-change and total expression.

% Using the transcript-wise estimated read counts from the simulated
% data, we performed DE analysis using \ebseq{}.  Of the 288
% differentially expressed transcripts, 179 (62\%) were correctly
% identified by \ebseq{}, with 29 false discoveries.

% %% In this example several of the DE transcripts were assigned a
% %% posterior probability of DE to be zero.  This is due to one of the
% %% treatment groups having zero count.  Also, some of the DE genes are
% %% removed by \ebseq{} because the read count is too low.  Thus some of
% %% the DE transcripts are unidentifiable.  If we remove them we have a
% %% true positive rate of $x$\%.

% Unaligned read data from twenty RNA-seq chips, along with a set of
% chromosome 22 reference files were used to simulate the reads in this
% example.  The example takes about 3 hours to run and may be found in
% the compendium package \compendium{}, available at
% \url{http://www.biostat.wisc.edu/~syounkin/}.
